\documentclass[12pt,t]{beamer}
\usepackage{graphicx}
\setbeameroption{hide notes}
\setbeamertemplate{note page}[plain]
\usepackage{listings}

% get rid of junk
\usetheme{default}
\beamertemplatenavigationsymbolsempty
\hypersetup{pdfpagemode=UseNone} % don't show bookmarks on initial view


% font
\usepackage{fontspec}
\setsansfont
  [ ExternalLocation = fonts/ ,
    UprightFont = *-regular ,
    BoldFont = *-bold ,
    ItalicFont = *-italic ,
    BoldItalicFont = *-bolditalic ]{texgyreheros}
\setbeamerfont{note page}{family*=pplx,size=\footnotesize} % Palatino for notes
% "TeX Gyre Heros can be used as a replacement for Helvetica"
% I've placed them in ../fonts/; alternatively you can install them
% permanently on your system as follows:
%     Download http://www.gust.org.pl/projects/e-foundry/tex-gyre/heros/qhv2.004otf.zip
%     In Unix, unzip it into ~/.fonts
%     In Mac, unzip it, double-click the .otf files, and install using "FontBook"

% named colors
\definecolor{offwhite}{RGB}{255,250,240}
\definecolor{gray}{RGB}{155,155,155}

\definecolor{background}{RGB}{255,255,255}
\definecolor{foreground}{RGB}{24,24,24}
\definecolor{title}{RGB}{27,94,134}
\definecolor{subtitle}{RGB}{22,175,124}
\definecolor{hilit}{RGB}{122,0,128}
\definecolor{vhilit}{RGB}{255,0,128}
\definecolor{lolit}{RGB}{95,95,95}

\definecolor{nhilit}{RGB}{128,0,128}  % hilit color in notes
\definecolor{nvhilit}{RGB}{255,0,128} % vhilit for notes

\newcommand{\hilit}{\color{hilit}}
\newcommand{\vhilit}{\color{vhilit}}
\newcommand{\nhilit}{\color{nhilit}}
\newcommand{\nvhilit}{\color{nvhilit}}
\newcommand{\lolit}{\color{lolit}}

% use those colors
\setbeamercolor{titlelike}{fg=title}
\setbeamercolor{subtitle}{fg=subtitle}
\setbeamercolor{institute}{fg=lolit}
\setbeamercolor{normal text}{fg=foreground,bg=background}
\setbeamercolor{item}{fg=foreground} % color of bullets
\setbeamercolor{subitem}{fg=lolit}
\setbeamercolor{itemize/enumerate subbody}{fg=lolit}
\setbeamertemplate{itemize subitem}{{\textendash}}
\setbeamerfont{itemize/enumerate subbody}{size=\footnotesize}
\setbeamerfont{itemize/enumerate subitem}{size=\footnotesize}

% page number
\setbeamertemplate{footline}{%
    \raisebox{5pt}{\makebox[\paperwidth]{\hfill\makebox[20pt]{\lolit
          \scriptsize\insertframenumber}}}\hspace*{5pt}}

% add a bit of space at the top of the notes page
\addtobeamertemplate{note page}{\setlength{\parskip}{12pt}}

% default link color
\hypersetup{colorlinks, urlcolor={hilit}}

\ifx\notescolors\undefined % slides
  % set up listing environment
  \lstset{language=bash,
          basicstyle=\ttfamily\scriptsize,
          frame=single,
          commentstyle=,
          backgroundcolor=\color{darkgray},
          showspaces=false,
          showstringspaces=false
          }
\else % notes
  \lstset{language=bash,
          basicstyle=\ttfamily\scriptsize,
          frame=single,
          commentstyle=,
          backgroundcolor=\color{offwhite},
          showspaces=false,
          showstringspaces=false
          }
\fi

% a few macros
\newcommand{\bi}{\begin{itemize}}
\newcommand{\bbi}{\vspace{24pt} \begin{itemize} \itemsep8pt}
\newcommand{\ei}{\end{itemize}}
\newcommand{\ig}{\includegraphics}
\newcommand{\subt}[1]{{\footnotesize \color{subtitle} {#1}}}
\newcommand{\ttsm}{\tt \small}
\newcommand{\ttfn}{\tt \footnotesize}
\newcommand{\figh}[2]{\centerline{\includegraphics[height=#2\textheight]{#1}}}
\newcommand{\figw}[2]{\centerline{\includegraphics[width=#2\textwidth]{#1}}}


%%%%%%%%%%%%%%%%%%%%%%%%%%%%%%%%%%%%%%%%%%%%%%%%%%%%%%%%%%%%%%%%%%%%%%
% end of header
%%%%%%%%%%%%%%%%%%%%%%%%%%%%%%%%%%%%%%%%%%%%%%%%%%%%%%%%%%%%%%%%%%%%%%

% title info
\title{Dissecting \emph{trans}-eQTL bands}
\author{Karl Broman \& Jianan Tian}
\institute{Biostatistics \& Medical Informatics, UW{\textendash}Madison}
\date{}


\begin{document}

% title slide
{
\setbeamertemplate{footline}{} % no page number here
\frame{
  \titlepage
}
}


\begin{frame}[c]{B6 $\times$ BTBR, \emph{Lep}$^{\text{\emph{ob}/\emph{ob}}}$}

\figw{TransBandsPaper/Figs/plot-eqtl.pdf}{1.1}

\end{frame}


\begin{frame}[c]{LDA \& PCA: Islet c2}
\figw{Figs/ldapca_islet2.pdf}{1.2}
\end{frame}

\begin{frame}[c]{LDA \& PCA: Islet c6}
\figw{Figs/ldapca_islet6.pdf}{1.2}
\end{frame}

\begin{frame}[c]{LDA \& PCA: Kidney c13}
\figw{Figs/ldapca_kidney13.pdf}{1.2}
\end{frame}

\begin{frame}[c]{LDA \& PCA: Liver c17}
\figw{Figs/ldapca_liver17.pdf}{1.2}
\end{frame}

\begin{frame}[c]{LDA \& PCA: Adipose c10}
\figw{Figs/ldapca_adipose10.pdf}{1.2}
\end{frame}



\begin{frame}[c]{eQTL effects: Kidney c13}
\figw{Figs/effects_kidney13.pdf}{1.2}
\end{frame}

\begin{frame}[c]{eQTL effects: Islet c6}
\figw{Figs/effects_islet6.pdf}{1.2}
\end{frame}

\begin{frame}[c]{eQTL effects: Islet c2}
\figw{Figs/effects_islet2.pdf}{1.2}
\end{frame}

\begin{frame}[c]{eQTL effects: Liver c17}
\figw{Figs/effects_liver17.pdf}{1.2}
\end{frame}

\begin{frame}[c]{eQTL effects: Adipose c10}
\figw{Figs/effects_adipose10.pdf}{1.2}
\end{frame}


\begin{frame}[c]{Formal test for 1 vs 2 QTL}

  \begin{itemize}
  \itemsep12pt
  \item Consider a set of traits mapping to common eQTL
  \item Multivariate QTL analysis with 1 or 2 QTL
  \item With 2-QTL model, each trait affected by one or the other QTL
    \vspace*{8pt}
    \begin{itemize}
      \itemsep8pt
      \item Order traits by est'd QTL location when considered
        separately
      \item Consider cut points of the list, assign first group to one
        QTL and second group to other.
    \end{itemize}
  \item P-value: parametric bootstrap or stratified permutation
  \end{itemize}

\end{frame}


\begin{frame}[c]{1 vs 2 QTL: Kidney c13}
\figw{Figs/formal_kidney13.pdf}{1.2}
\end{frame}

\begin{frame}[c]{1 vs 2 QTL: Islet c6}
\figw{Figs/formal_islet6.pdf}{1.2}
\end{frame}

\begin{frame}[c]{1 vs 2 QTL: Islet c2}
\figw{Figs/formal_islet2.pdf}{1.2}
\end{frame}

\begin{frame}[c]{1 vs 2 QTL: Liver c17}
\figw{Figs/formal_liver17.pdf}{1.2}
\end{frame}

\begin{frame}[c]{1 vs 2 QTL: Adipose c10}
\figw{Figs/formal_adipose10.pdf}{1.2}
\end{frame}


\begin{frame}[c]{Summary}
  \begin{itemize}
  \itemsep18pt
  \item Exploratory plots
    \vspace*{12pt}
    \begin{itemize}
    \itemsep12pt
    \item LDA or PCA results
    \item Sign of eQTL effect
    \item Degree of dominance
    \end{itemize}
  \item Formal test of 1 vs 2 QTL
  \end{itemize}
\end{frame}

\end{document}
